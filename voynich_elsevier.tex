
\documentclass[review]{elsarticle}

\usepackage{graphicx}
\usepackage{hyperref}
\usepackage{amsmath}

\begin{document}

\begin{frontmatter}

\title{Ergotism, Cognitive Distortion, and Women's Medicine:\\
An Interdisciplinary Explanation for the Voynich Manuscript}

\author{Thomas Couch}

\begin{abstract}
This article presents a unified hypothesis proposing that the Voynich Manuscript was produced by an individual undergoing
ergotism-induced cognitive distortion. The structured but nonsensical text, hybrid botanical drawings, and women's medical imagery 
fit a model where a trained scribe's procedural memory remains intact while semantic processing collapses under toxic influence.
\end{abstract}

\end{frontmatter}

\section{Introduction}
The Voynich Manuscript has resisted decipherment for more than a century...

\section{Environmental and Historical Context}
Cold and wet conditions from 1404--1438 increased ergot infections...

\section{Cognitive Mechanisms}
Ergot alkaloids induce linguistic impairment analogous to aphasia...

\section{Manuscript Comparisons}
\includegraphics[width=0.8\linewidth]{manuscript_schema.png}

\section{Entropy Analysis}
\includegraphics[width=0.6\linewidth]{entropy_detailed.png}

\section{Conclusion}
The ergotism hypothesis explains...

\bibliographystyle{elsarticle-num}
\bibliography{references}

\end{document}
